% 
% (c) 2015, Florian Mayer
%
% This document is released under the terms
% of the GNU Free Documentation License
%
% See LICENSE for further information regarding
% this license.
%%

\documentclass[11pt,a4paper]{article}
\usepackage{listings,palatino,avant,graphicx,color,pslatex}
\usepackage[ngerman]{babel}
\usepackage[margin=2cm]{geometry}
\usepackage{courier}
\usepackage[table]{xcolor}
\usepackage[utf8]{inputenc}
\usepackage{graphicx}
\usepackage{lmodern}
\usepackage{amsmath}
\usepackage{bytefield}
\usepackage[colorlinks=true,linkcolor=lightblue,citecolor=black,urlcolor=lightblue]{hyperref}

\definecolor{lightblue}{rgb}{0.2925,0.6381,0.65}
\definecolor{darkblue}{rgb}{0.0105,0.2764,0.35}
\definecolor{grey}{gray}{0.75}

\lstset{numbers=none,numberstyle=\small\ttfamily,stepnumber=1,numbersep=4pt}
\lstset{tabsize=4}
\lstset{breaklines=true, breakatwhitespace=true}
\lstset{frame=none}

\lstdefinestyle{c}
	{language=C, identifierstyle=\color{darkblue},
	 basicstyle=\small\ttfamily, keywordstyle=\color{lightblue}\bfseries,
	 commentstyle=\color{grey}}

\lstdefinestyle{bash}
	{basicstyle=\small\ttfamily,language=bash, 
	 identifierstyle=\color{darkblue}}

\def\inlinebash{\lstinline[style=bash]}
\def\inlinec{\lstinline[style=c]}

\begin{document}
\title{\color{black} Ext4-Übung}
\author{\color{darkblue} Florian Mayer}
\maketitle

\tableofcontents

\section{Einführung}
Im ersten Kapitel dieses Dokuments werden die 
Werkzeuge des e2fsprogs-Pakets näher beleuchtet.
Danach folgt die eigentliche Übung

\subsection{Bestandteile des e2fsprogs-Pakets}
Die Tools -- soweit nicht anders vermerkt -- arbeiten 
jeweils mit ext2, ext3 oder ext4 Dateisystemen. ``ExtX'' bezeichnet
zusammenfassend ext2- ext3- oder ext4-Systeme.

\paragraph{e2fsk}
	Überprüft ein ExtX-System auf Inkonsistenzen und behebt diese, wenn möglich.

\paragraph{mke2fs}
	Wird vom Frontendprogramm \inlinebash$mkfs$ verwendet um neue ExtX-Dateisysteme anzulegen.

\paragraph{resize2fs}
	Kann genutzt werden, um ein ExtX-Dateisystem an eine gewachsene oder
	verkleinerte Partition anzupassen.

\paragraph{tune2fs}
	Setzt oder modifiziert Dateisystemparameter.

\paragraph{dumpe2fs}
	Schreibt Superblock- bzw. Blockgruppeninformationen auf die Standardausgabe.

\paragraph{filefrag}
	Zeigt den Grad der Dateifragmentierung an.

\paragraph{e2label}
	Verändert das Dateisystemlabel eines ExtX-Systems.

\paragraph{findfs}
	Sucht nach einem Dateisystem mit einem Label oder einer UUID.

\paragraph{e2freefrag}
	Wie \inlinebash$filefrag$, jedoch mit dem Unterschied, dass nur die Fragmentierung
	des freien Speicherplatzes aufgezeigt wird.

\paragraph{chattr}
	Ähnlich zu \inlinebash$chmod$ oder \inlinebash$chown$. Funktioniert auf vielen verschiedenen Dateisystemen. 

\paragraph{e2image}
	Speichert kritische ExtX-Metadaten in einer Datei. 

\paragraph{e4defrag}
	Defragmentiert ein Ext4-Dateisystem während es parallel
	weiterhin genutzt werden kann.

\paragraph{findsuper}
	Findet Ext2-Superblocks. Veraltet und lt. Dokumentation ``schnell zusammengehackt''.

\subsection{Verwendete Programme}
Im Rahmen dieses Dokuments werden lediglich die Programme \inlinebash$debugfs$
und \inlinebash$dumpe2fs$ verwendet. Die Manualseiten könnten, wie gewohnt, mit
\begin{lstlisting}[style=bash]
$ man debugfs
$ man dumpe2fs
\end{lstlisting}
angesehen werden. Das Studium dieser Handbücher ist dringend empfohlen.

\section{Übung}

\subsection{Vorbereitungen des Test-Dateisystems}

Operationen auf Dateisysteme mit sog. Debugging-Kommandos
sind gefährlich. Löscht man z.B. versehentlich einen Inode, so
ist es nur noch schwer möglich die zugehörige Datei wiederherzustellen. 
Es wird daher beschrieben, wie eine sog.
Sandbox-Umgebung (also ein vom System abgekapseltes Dateisystem)
eingerichtet werden kann. Die hier geschilderte Methode steht
sofort unter GNU/Linux zur Verfügung, da sie integraler Kernel-Bestandteil ist.

\begin{itemize}
\item \inlinebash$dd if=/dev/zero of=~/foo.disk count=1000$\\
	Erzeugt eine genullte Datei, bestehend aus 1000 Blöcken zu jeweils 0,5KiB.

\item \inlinebash$losetup --list$\\
	Zeigt bereits vorhandene loop devices (footnote) an. Bereits vorhandene Loop-Geräte
	müssen mittels \inlinebash$sudo losetup -d /dev/loop<X>$ ausgehängt werden. 

\item \inlinebash$sudo losetup /dev/loop0 ~/foo.disk$\\
	Erzeugt eine blockorientierte Gerätedatei, die alle Lese- und Schreiboperationen
	auf die Datei \~/foo.disk abbildet. Das Kommando muss sodann mithilfe von
	\inlinebash$losetup --list$ überprüft werden. Der Vorgang war erfolgreich, 
	wenn nun die neue Konfiguration ausgegeben wird.

\item \inlinebash$sudo parted /dev/loop0 mklabel msdos$\\
	Erstellt eine Partitionstabelle mit \emph{msdos}-Format. Das bedeutet, dass
	ein normaler MBR erstellt wird.
	
\item \inlinebash$sudo parted /dev/loop0 mkpart primary 0 500K$\\
	Das erste Kommando erzeugt einen neuen MBR und das zweite
	richtet eine primäre Partition über die volle Länge des
	virtuellen Blockgeräts ein.

\item \inlinebash$sudo mkfs.ext4 /dev/loop0p1$\\
	Erzeugt das zu untersuchende Dateisystem. Achtung: Hier darf
	/dev/loop0 nicht verwendet werden, da dies keine Partition ist!
	Die Gerätedatei der Partition (\emph{/dev/loop0p1}) wird automatisch
	vom System erkannt und steht daher sofort nach Erzeugung der
	Partition zur Verfügung. Hinweis: Merken Sie sich die Ausgabe des
	Kommandos! Sie werden Sie in einer der Übungen benötigen!

\item
Nun muss das frische Dateisystem nur noch eingebunden werden:

\begin{lstlisting}[style=bash]
$ sudo mkdir /mnt/foo
$ sudo mount /dev/loop0p1 /mnt/foo
\end{lstlisting}
\end{itemize}

\subsection{Allgemeine Fragen}
\begin{enumerate}
\item Ab welchem Offset beginnt der Superblock und
	wie groß ist er? Hinweis: Der Superblock beginnt typischerweise unmittelbar
	nach dem nullten Block, also im ersten (vgl. hierzu Kapitel 2.2 ``Layout''
	\href{https://ext4.wiki.kernel.org/index.php/Ext4_Disk_Layout#Layout}{der offiziellen Dokumentation})
\item Wieviel Inodes und Datenblöcke wurden beim Erstellen des Dateisystems erzeugt?
	Hinweis: Die Ausgabe des \inlinebash$mkfs.ext4$ Kommandos hilft. Alternativ 
	kann auch \inlinebash$dumpe2fs$ benutzt werden.
\item Wie groß ist ein Ext4-Datenblock und wie viele sind davon anfangs
	tatsächlich durch Dateien nutzbar? Hinweis: Nutzen Sie \inlinebash$dumpe2fs$ und
	suchen Sie nach dem Feld ``Block size''!
\item Wieviele Bytes ist ein Inode groß? Hinweis: Nutzen Sie wieder \inlinebash$dumpe2fs$
\end{enumerate}

\subsection{Dumpe2fs}
	Beantworten Sie mithilfe des Programms \inlinebash$dumpe2fs$ 
	die folgenden Fragen unter der Annahme,
	dass Sie sich auf einem \emph{32-Bit-System}
	befinden\footnote{Tatsächlich werden Sie jedoch auf 
		einem 64-Bit-System arbeiten}. (Verwenden Sie für diese
	Übung Kapitel 2.1
	\href{https://ext4.wiki.kernel.org/index.php/Ext4_Disk_Layout#Blocks}{der offiziellen ext4-Dokumentation}. Die relevanten Daten finden in der Tabelle ``File System Maximums'')
	\begin{enumerate}
		\item Wie groß kann das Dateisystem maximal werden? 
			Hinweis: Multiplizieren Sie die Blockgröße (vgl. Lösung der Aufgabe 2.2.3)
			mit der maximalen Anzahl der Blöcke. Beachten Sie die Wortgröße!
		\item Wie viele Inodes kann es maximal geben? Hinweis: 
			Jeder Inode hat (unter x86) eine 4-Byte-Inode-Nummer und keine zwei Inodes
			dürfen dieselbe Inode-Nummer haben.
		\item Wie groß kann eine Datei, die in eine einzige Block-Gruppe passt,
			maximal sein? Hinweis: Die Ausgabe von \inlinebash$dumpe2fs$ liefert
			einen Eintrag, der auf die Menge der Datenblöcke innerhalb einer
			Blockgruppe Aufschluss gibt. Nutzen Sie diesen Wert und erinnern Sie sich
			an die Blockgröße. 
	\end{enumerate}

\subsection{Einlesen des Superblocks}
Schreiben Sie ein C-Programm mit dem sie den Superblock einlesen und
einige Informationen auf der Kommandozeile ausgeben. 

\begin{itemize}
	\item Anzahl der Inodes
	\item Insgesamte Anzahl der Blöcke
	\item Erster Datenblock
	\item Anzahl freier Inodes
	\item Anzahl freier Datenblöcke
	\item Magic Number (Ausgabe in Hex)
	\item 128-bit UUID des Dateisystems (Ausgabe in Hex)
	\item Betriebssystem des Erstellers
\end{itemize}

\begin{lstlisting}[style=c]
/* Blaupause fuer die Loesung */
#include <ext2fs/ext2fs.h>
#include <ext2fs/ext2_fs.h>
#include <stdio.h>
#include <stdlib.h>
#include <et/com_err.h>
#include <stdint.h>

static const char DEV[] = "/dev/loop0p1";
static const char OS_L[] = "LINUX";
static const char OS_O[] = "OTHER";

int main(int argc, char **argv){
	errcode_t err;
	ext2_filsys fs;
	struct ext2_super_block *super;

	/* Oeffnen des Dateisystemhandles "fs" mithilfe der Funktion ext2fs_open.
		Hinweis: Verwenden Sie Kapitel 4 als Referenz und nutzen Sie
		EXT2_FLAG_RW, 0, 0, unix_io_manager als Argument 2 - 5 fuer die Funktion!
		Argument 1 und 6 muessen Sie selbst herausfinden. */
	??? 

	if (err){
		if (err == 13) {
			printf("[ext2fs_open] Insufficient permissions\n");
		} else {
			printf("[ext2fs_open] Error number: %ld\n", err);
		}
		return EXIT_FAILURE;
	}
	
	super = fs->super;	

	/* Hinweis: Hier printf-Statements der folgenden Member von super:
		super->s_inodes_count (integer)
		super->s_blocks_count (integer)
		super->s_first_data_block (integer)
		super->s_free_inodes_count (integer)
		super->s_free_blocks_count (integer)
		super->s_magic (%02X)
		super->s_uuid[] (verwenden Sie *(((uint32_t *) super->s_uuid) + <Index>)
		super->s_creator_os 
			(integer. Geben Sie "Linux" aus, 
			wenn Wert gleich 0, sonst "Other") */
	???

	/* Schliessen des Dateisystem-Handles mittels ext2fs_close(fs). 
		Nuten Sie die in Kapitel 4 dokumentierte Funktion ext2fs_close(ext2_filsys FS)*/
	???

	if (err){
		printf("[ext2fs_close] Error number: %ld\n", err);
		return EXIT_FAILURE;
	}
	return EXIT_SUCCESS;
}
\end{lstlisting}

\subsection{Anlegen von Testdaten}

\begin{enumerate}
\item Erstellen Sie im neuen Dateisystem den angegebenen Verzeichnisbaum 
	Hierfür steht ihnen ein Bash-Skript zur Verfügung (vgl. scripts/file\_tree.sh)
	Der Aufruf lautet \inlinebash$sudo ./file_tree.sh$. Das Skript wechselt automatisch
	ins richtige Verzeichnis und wieder zurück. Der folgende Baum wird generiert:

\begin{verbatim}
/+-> t1
 |   +-> dirA
 |      +-> foo.txt  -- Inhalt: "FOO.TXT"
 |   +-> a.txt -- Inhalt "A"
 |   +-> b.txt -- Inhalt "B" 
 |   +-> c.txt -- 2048 mal "abc "
 +-> t2
 |   +-> dirB
 |   +-> dirC
 +-> foo.txt -- Inhalt "FOO"
 
\end{verbatim}
\end{enumerate}

\subsection{Debugfs}
In dieser Aufgabe machen Sie sich mit dem Dateisystemdebugger
\inlinebash$debugfs$ vertraut.

\begin{enumerate}
	\item Starten Sie debugfs (\inlinebash$sudo debugfs /dev/loop0p1$).
	\item Welcher Inode stellt das Verzeichnis ``t1'' dar, welcher ``t2''?
		(Hinweis \inlinebash$ls -l$\footnote{
			Es handelt sich hierbei um ein Kommando von \emph{debugfs}, ist also
			insbesondere nicht mit dem \emph{ls}-Kommando zu verwechseln.})
	\item Welcher Inode stellt die Verzeichniswurzel dar und
		auf welchem Block (i.S.v. Ext4) liegt dieser?
		(Hinweis \inlinebash$imap <inode nummer> ODER <pfad>$)
	\item Wechseln Sie mit \inlinebash$cd$ in das Verzeichnis ``t1'' und lassen
		Sie sich alle Inodes anzeigen. Benutzen Sie das ``cat''-
		Kommando, um sich den Inhalt der Dateien a.txt und b.txt
		ausgeben zu lassen.
	\item Wo liegt der Block des Inodes für a.txt?
	\item Lassen Sie sich mittels \inlinebash$stat <inode nummer>$ den
		Inodestatus für die Datei a.txt ausgeben.
		Auf welchem Block liegen die Daten?
	\item Geben Sie den Datenblock mittels \inlinebash$block_dump$ aus.
	\item Wie viele Blocks benötigt die Datei c.txt?
		(Hinweis: \inlinebash$blocks <inode nummer>$)
	\item Geben sie auch noch einen der ausgegebenen Blöcke
		auf der Standardausgabe aus (\inlinebash$block_dump$).
\end{enumerate}

\subsection{Inodes}
\begin{enumerate}
	\item Lassen Sie sich den Inode der Datei \emph{a.txt} mithilfe der Kommandos \inlinebash$dd$
		und \inlinebash$hexdump$ anzeigen. Hinweis: Suchen Sie zunächst die Nummer des Inodes.
		Nun können Sie die Blocknummer des Inodes und dessen Offset zu diesem Block ermitteln.
		Erinnern Sie sich auch daran, dass ein Inode typischerweise 128 Byte groß sein wird!
	\item Schreiben Sie ein C-Programm mit dem Sie sich die folgenden Daten
		des Inodes der Datei \emph{c.txt} ausgeben lassen. Nutzen Sie dazu die angegebene
		Vorlage und Kapitel 4.3 und achten Sie auf die Fragezeichen im Code.
		\begin{itemize}
			\item User-ID
			\item Modifikationsdatum (nur als Integer ausgeben)
			\item Zugriffsdatum (nur als Integer ausgeben)
			\item Berechtigungen (Dateimodus)
			\item Anzahl der Blöcke in Bytes
		\end{itemize}
		
		\paragraph{Blaupause für die Lösung}
		\begin{lstlisting}[style=c]
#include <ext2fs/ext2fs.h>
#include <ext2fs/ext2_fs.h>
#include <stdio.h>
#include <stdlib.h>
#include <et/com_err.h>
#include <stdint.h>

static const char DEV[] = "/dev/loop0p1";
static const int INODE = ???;

int main(int argc, char *argv[]){
	errcode_t err;
	struct struct_ext2_filsys *fs;
	struct ext2_inode inode;

	err = ext2fs_open(DEV, EXT2_FLAG_RW, 0, 0, unix_io_manager, &fs);
	if (err){
		if (err == 13) {
			printf("[ext2fs_open] Insufficient permissions\n");
		} else {
			printf("[ext2fs_open] Error number: %ld\n", err);
		}
		return EXIT_FAILURE;
	}

	???
	err = ext2fs_read_inode(fs, INODE, &inode);
	if (err){
		printf("[ext2fs_read_inode] Could not read inode %d\n", INODE);
		return EXIT_FAILURE;
	}

	/* Hier printf-Statements einfuegen. 
		inode.i_uid -> User-id (integer)
		inode.i_mtime -> Modifikationsdatum (integer)
		inode.i_atime -> Zugriffsdatum (integer)
		inode.i_mode ->Berechtigungen (integer, oktal!)
		inode.i_blocks -> Blockanzahl (integer) */
	???

	err = ext2fs_close(fs);
	if (err){
		printf("[ext2fs_close] Error number: %ld\n", err);
		return EXIT_FAILURE;
	}

	return EXIT_SUCCESS;
}

		\end{lstlisting}
\end{enumerate}

\section{Blockgruppen Layout von Ext4}
Die folgende Tabelle beschreibt den grundsätzlichen Aufbau einer ext4-Blockgruppe.
Die Menge der Blöcke, die in einer Blockgruppe enthalten sind, lässt sich leicht
mithilfe des \inlinebash$dumpe2fs$-Kommandos ermitteln (vgl. auch 2.3).
\paragraph{}

\begin{table}[h]
	\begin{center}
		\begin{tabular}[c]{| l | l |}
			\hline
			\cellcolor{grey} Daten & \cellcolor{grey} Größe \\ \hline
			Füllbytes vor der Gruppe 0 & 1024 Bytes \\ \hline
			Super Block & 1 Block (1KiB, 2KiB, 4KiB, \ldots, 64KiB) \\ \hline
			Group-Descriptors & n Blöcke  \\ \hline
			Reservierte Group-Descriptors & k Blöcke. Werden für's nachträgliche Vergrößern benötigt \\ \hline
			Datenblock Bitmap & 1 Block \\ \hline
			Inode Bitmap & 1 Block \\ \hline
			Inode Tabelle & j Blöcke (Im Test-Dateisystem 8 1KiB-Blöcke) \\ \hline
			Datenblöcke & Der Rest \\ \hline
		
		\end{tabular}
	\end{center}

	\caption{Blockgruppen Layouttabelle}
	\label{tab:block_layout}
\end{table}

\section{Ext4-API-Referenz}
Im Folgenden einige Auszüge aus der offiziellen API-Referenz für das
Ext4-Dateisystem.

\subsection{Öffnen eines ext4-Dateisystems}
\begin{verbatim}
Most libext2fs functions take a filesystem handle of type 'ext2_filsys'.
A filesystem handle is created either by opening an existing function
using 'ext2fs_open', or by initializing a new filesystem using
'ext2fs_initialize'.

-- Function: errcode_t ext2fs_open (const char *NAME, int FLAGS, int
SUPERBLOCK, int BLOCK_SIZE, io_manager MANAGER, ext2_filsys
*RET_FS)

Opens a filesystem named NAME, using the the io_manager MANAGER to
define the input/output routines needed to read and write the
filesystem.  In the case of the 'unix_io' io_manager, NAME is
interpreted as the Unix filename of the filesystem image.  This is
often a device file, such as '/dev/hda1'.

The SUPERBLOCK parameter specifies the block number of the
superblock which should be used when opening the filesystem.  If
SUPERBLOCK is zero, 'ext2fs_open' will use the primary superblock
located at offset 1024 bytes from the start of the filesystem
image.

The BLOCK_SIZE parameter specifies the block size used by the
filesystem.  Normally this is determined automatically from the
filesystem uperblock.  If BLOCK_SIZE is non-zero, it must match the
block size found in the superblock, or the error
'EXT2_ET_UNEXPECTED_BLOCK_SIZE' will be returned.  The BLOCK_SIZE
parameter is also used to help fund the superblock when SUPERBLOCK
is non-zero.

The FLAGS argument contains a bitmask of flags which control how
the filesystem open should be handled.

'EXT2_FLAG_RW'
Open the filesystem for reading and writing.  Without this 
flag, the filesystem is opened for reading only.

'EXT2_FLAG_FORCE'
Open the filesystem regardless of the feature sets listed in
the superblock.
\end{verbatim}

\subsection{Schließen eines Ext4-Dateisystems}
\begin{verbatim}
-- Function: errcode_t ext2fs_flush (ext2_filsys FS)

Write any changes to the high-level filesystem data structures in
the FS filesystem.  The following data structures will be written
out:

* The filesystem superblock
* The filesystem group descriptors
* The filesystem bitmaps, if read in via 'ext2fs_read_bitmaps'.

-- Function: void ext2fs_free (ext2_filsys FS)

Close the io_manager abstraction for FS and release all memory
associated with the filesystem handle.

-- Function: errcode_t ext2fs_close (ext2_filsys FS)

Flush out any changes to the high-level filesystem data structures
using 'ext2fs_flush' if the filesystem is marked dirty; then close
and free the filesystem using 'ext2fs_free'.

\end{verbatim}

\subsection{Lesen eines Inodes}
\begin{verbatim}
-- Function: errcode_t ext2fs_read_inode (ext2_filsys FS, ext2_ino_t
INO, struct ext2_inode *INODE)
Read the inode number INO into INODE.

-- Function: errcode_t ext2fs_write_inode (ext2_filsys FS, ext2_ino_t
INO, struct ext2_inode *INODE)
Write INODE to inode INO. 

\end{verbatim}

\end{document}
